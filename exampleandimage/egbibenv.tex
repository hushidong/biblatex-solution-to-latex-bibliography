\documentclass[twoside]{report}
\usepackage{ctex}
\usepackage{geometry}
\geometry{paperwidth=15cm,paperheight=15cm,top=0cm, bottom=1.5cm, left=1cm,right=1cm}
\usepackage{titlesec}
\titleformat{\chapter}{\zihao{4}\heiti}{\thechapter}{1em}{}
\titlespacing*{\chapter}{0pt}{0.0\baselineskip}{0.5\baselineskip}[0pt]
\titleformat{\section}{\centering\bfseries}{第\thesection 节}{1em}{}[]
\titlespacing*{\section}{0pt}{0.0\baselineskip}{0.0\baselineskip}[0pt]
\titleformat{\subsection}{\flushleft\bfseries}{\S\,\thesubsection}{1em}{}[]
\titlespacing*{\subsection}{0pt}{0.0\baselineskip}{0.0\baselineskip}[0pt]
\usepackage{xcolor}
%书签功能,选项去掉链接红色方框
\usepackage[CJKbookmarks,colorlinks,bookmarksnumbered=true,pdfstartview=FitH,linkcolor=blue]{hyperref}
\usepackage[backend=biber,style=gb7714-2015]{biblatex}
\addbibresource[location=local]{example.bib}

\makeatletter
\def\blx@bibitem#1{%
  \blx@ifdata{#1}
    {\begingroup
     \blx@getdata{#1}%
     \blx@bibcheck
     \iftoggle{blx@skipentry}{}{%
       \blx@setdefaultrefcontext{#1}%
       \global\let\blx@noitem\@empty
       \blx@setoptions@type\abx@field@entrytype
       \blx@setoptions@entry
       \blx@thelabelnumber
       \addtocounter{instcount}\@ne
       \blx@initsep
       \blx@namesep
       \csuse{blx@item@\blx@theenv}\relax
%       \blx@initsep   %移动到上面去,恢复bibnamesep等的作用机制
%       \blx@namesep
       \csuse{blx@hook@bibitem}%
       \blx@execute
       \blx@initunit
       \blx@anchor
       \blx@beglangbib
       \bibsentence
       \blx@pagetracker
       \blx@driver\abx@field@entrytype
       \blx@postpunct
       \blx@endlangbib}%
     \par\endgroup}%这里增加了一个\par
    {}}
\makeatother
\newcommand{\itemcmd}{%
\stepcounter{bibentrynumber}
\settowidth{\lengthid}{[\arabic{bibentrynumber}]}
\addtolength{\lengthid}{\biblabelsep}
\setlength{\lengthlw}{\textwidth}
\addtolength{\lengthlw}{-\lengthid}
\addvspace{\bibitemsep}%恢复\bibitemsep的作用
%\parshape 2 0em \textwidth \lengthid \lengthlw
\hangindent\lengthid
[\arabic{bibentrynumber}]\hspace{\biblabelsep}}
\newcounter{bibentrynumber}
\newlength{\lengthid}
\newlength{\lengthlw}
\defbibenvironment{envtest}
{\begingroup\setlength{\parindent}{0em}\setcounter{bibentrynumber}{0}}
{\endgroup}
{\itemcmd}%\newline\itemcmd
%
%参考文献文本字体设置为默认,字号为6,利用ctex设置
%如果不是利用ctex宏包,可以利用其它字号设置命令
\renewcommand{\bibfont}{\zihao{5}}
%设置各条参考文献之间的间距为0pt
\setlength{\bibitemsep}{0ex}
\setlength{\bibnamesep}{0ex}
\setlength{\bibinitsep}{0ex}


\newcommand{\itemcmdb}{\addvspace{\bibitemsep}}
\defbibenvironment{marginref}{\begingroup\setlength{\parindent}{0em}}
{\endgroup}{\itemcmdb}%\newline%\@endparenv%\clearpage

\defbibenvironment{bibliography}
{\list
{\printfield[labelnumberwidth]{labelnumber}}%标签,由域labelnumber信息的提供
{\setlength{\labelwidth}{\labelnumberwidth}%标签宽度设置为labelnumberwidth
\setlength{\leftmargin}{\labelwidth}%
\setlength{\labelsep}{\biblabelsep}%标签与内容间距设置为\biblabelsep
\addtolength{\leftmargin}{\labelsep}%缩进宽度设置为\labelwidth+\labelsep
\setlength{\itemsep}{\bibitemsep}%垂直间距
\setlength{\parsep}{\bibparsep}}%
\renewcommand*{\makelabel}[1]{\hss##1\hfill}}
{\endlist}
{\item}

\begin{document}
\chapter{文献表自定义}
\begin{refsection}
\section{文献表自定义}

{\hangindent=2em \par She very soon came to an open field, with
a wood on the other side of it: it looked much darker
than the last wood, and Alice felt a little timid
about going into it.\par}

{\hangindent=2em She very soon came to an open field, with
a wood on the other side of it: it looked much darker
than the last wood, and Alice felt a little timid
about going into it.\par}

{\hangindent=2em She very soon came to an open field, with
a wood on the other side of it: it looked much darker
than the last wood, and Alice felt a little timid
about going into it.}

 She very soon came to an open field, with
a wood on the other side of it: it looked much darker
than the last wood, and Alice felt a little timid
about going into it.

序章内容\cite{GPS1988--,杨洪升2013-56-75,马克思2013-302-302}
正文内容一\cite{Andersen1995-42-49,BUSECK1980-117-211,Calkin2011-8-9}
正文内容二\cite{Parsons2000b--,Parsons2000--,Parsons2000noloc--,Parsons2000nodate--}
\cite{1977-49-49,亚洲地质图编目组1978-194-208,陈晋镳1980-56-114}



\printbibliography[heading=subbibintoc,env=envtest,title=【参考文献】]
\end{refsection}

\chapter{文献表自定义}
\begin{refsection}
\section{文献表自定义}

序章内容\cite{GPS1988--,杨洪升2013-56-75,马克思2013-302-302}
正文内容一\cite{Andersen1995-42-49,BUSECK1980-117-211,Calkin2011-8-9}
正文内容二\cite{Parsons2000b--,Parsons2000--,Parsons2000noloc--,Parsons2000nodate--}
\cite{1977-49-49,亚洲地质图编目组1978-194-208,陈晋镳1980-56-114}

\printbibliography[heading=subbibintoc,env=marginref,title=【参考文献】]
\end{refsection}

\chapter{文献表自定义}
\begin{refsection}
\section{文献表自定义}

序章内容\cite{GPS1988--,杨洪升2013-56-75,马克思2013-302-302}
正文内容一\cite{Andersen1995-42-49,BUSECK1980-117-211,Calkin2011-8-9}
正文内容二\cite{Parsons2000b--,Parsons2000--,Parsons2000noloc--,Parsons2000nodate--}
\cite{1977-49-49,亚洲地质图编目组1978-194-208,陈晋镳1980-56-114}

\printbibliography[heading=subbibintoc,title=【参考文献】]
\end{refsection}
\end{document} 