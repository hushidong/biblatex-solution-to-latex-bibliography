\documentclass{article}
\usepackage{ctex,hyperref}
\usepackage{geometry}
\geometry{paperwidth=21cm,paperheight=26cm,%
left=1cm,right=1cm,top=1cm,bottom=1.5cm}
\usepackage[backend=biber,style=numeric,sorting=none,defernumbers=true]{biblatex}%gb7714-2015,defernumbers=true
\addbibresource{example.bib}
\renewcommand{\bibfont}{\zihao{6}}
\usepackage{titlesec}
%\titleformat{command}[shape]{format}{label}{sep}{before}[after]
\titleformat{\section}{\centering\bfseries}{第\thesection 节}{1em}{}[]
\titlespacing*{\section}{0pt}{0.0\baselineskip}{0.0\baselineskip}[0pt]
\titleformat{\subsection}{\flushleft\bfseries}{\S\,\thesubsection}{1em}{}[]
\titlespacing*{\subsection}{0pt}{0.0\baselineskip}{0.0\baselineskip}[0pt]
\begin{document}

\small 	参考文献测试\cite{Gradshteyn2000--}。

	\begin{refsegment}
		\section{refSegment A}
		分章节参考文献测试\cite{Chiani2003-840-845}
        %\printbibliography[heading=subbibliography,title=文献A] %不制定segment则遍历所有的segment
		\printbibliography[segment=1,heading=subbibliography,title=文献A]
	\end{refsegment}
	
	\begin{refsegment}
		\section{refSegment B}
		参考文献测试\cite{张敏莉2007-500-503}
	\end{refsegment}
	%\printbibliography放在refsegment环境外也是可以的
	\printbibliography[segment=2,heading=subbibliography,title=文献B]
	
	\begin{refsegment}
		\section{refsegment C}
		分章节参考文献测试\cite{Simon2004--}。
	\end{refsegment}

	\begin{refsegment}
		\section{refsegment D}
		分章节参考文献测试\cite{Lin2004--}。
	\end{refsegment}

	%遍历非refsection内的参考文献
%	\printbibliography[heading=bibliography,title=文献全局]
    
%    \begin{refcontext}
%    \printbibliography[heading=bibliography,title=文献全局,resetnumbers=true]
%    \end{refcontext}

%    \printbibliography[heading=bibliography,title=文献全局,sorting=none]

    \begin{refcontext}[sorting=none]
    \printbibliography[heading=bibliography,title=文献全局,resetnumbers=1]
    \end{refcontext}


\end{document} 